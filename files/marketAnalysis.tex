\chapter{Market analysis}
This chapter means to examine the market for products similar to character too our own.

\section{SOTX}
A company names SOTX have made a Smart Badminton Racquet. This racquet is labelled as having sensors and a microprocessor built into the handle.  The racuqet can then communicate with and Android or iOS smartphones via Bluetooth. It supports both live streaming aswell as transfering data at the end of a match. It seems to come with a racquet charger you can plug your racquet into. But it advertises wireless charging, how those two things work together isn't documented. 

From the look of the app, it seems like it focus mostly on what kind of swing you perform. But can also keep track on how long you've worked out, calories burned and max speed

\section{Holy Pie Smart Racket}
In 2014 a kickstarter campaign called Holy Pie Smart Racket launched. It succesfully raised 2010 dollars with 4 backers. They made a system for both badminton and tennis. It's tough to tell what this product really can do for you since their website is really bad and mostly in chinese and seems very unfinished as most of the links does not work and there is still dummy text present on the site.

\section{USENSE}
Usense is a sensor that you mount at the bottom of the racket shaft. There is an android and iOS app for the sensor so that it can connect to your phone through bluetooth 4.0+. This one is significantly different from the other two because you dont need to buy a new badminton racket. You only buy a sensor and mount it. The downside by it not being integrated into the racket is the additional weight it adds to a place where you do not want additional weight.

\section{Reflection}
All of these products seem very undocumented and their websites are not very user friendly. Especially not the English language ones. USENSE is basically just a product on Aliexpress.com. They are only avaliable through a phone application and no where else. 