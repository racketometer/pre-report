\chapter{Introduction}

In some sports the performance measurement is very subjective with terms like ''good force'',
''steady hand'' etc. This is not measurable and could be improved upon. This project is centered
around athletes and the goal is to enable performance tracking and bench marking with
quantitative data. This has potential world wide and therefore it could be beneficial to support
multiple platforms.

\subsection*{Data analysis}
An external sensor, delivered by an external collaborator, will supply detailed measurements of the
athletes’ movements and activities. This data is not sensible to a user and needs to be analyzed
so that key features can be extracted. The graphs, figure \ref{fig:graph:elite_novice}, show measured data from the sensors
during a badminton smash, where the sensor is mounted inside the racket. The one to the left is
an average person and the one to the right is a professional athlete.

\graphic{1}{graph_novice_elite}{Novice player to the left. Elite player to the right}{fig:graph:elite_novice}

\subsection*{Data presentation}
A part of the platform will focus on a sensible presentation of this data to the users. The data will
be presented with graphics and stats. The application should enable the individual athlete to
visualize the performance history and share personal records and alike on social media.
Business cases
\begin{itemize}
	\item The data from the sensor can be used by salesmen to determine which badminton rack is most suitable for a specific athlete.
	\item Personal performance monitoring for the individual athlete.
	\item Rehabilitation – monitoring your performance before and after injuries to determine if you’ve recovered fully.
	\item Consultant workshops – Help athletes improve their swings based on the data
\end{itemize}
